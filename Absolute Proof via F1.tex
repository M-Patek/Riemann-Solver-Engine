\documentclass[11pt, a4paper]{article}

% --- PREAMBLE ---
\usepackage[a4paper, top=2.5cm, bottom=2.5cm, left=2.5cm, right=2.5cm]{geometry}
\usepackage{fontspec}
\usepackage[english, provide=*]{babel}

% Font Setup: Standard academic pairing
\babelfont{rm}{Noto Serif}
\babelfont{sf}{Noto Sans}

% --- MATH PACKAGES ---
\usepackage{amsmath}
\usepackage{amssymb}
\usepackage{amsthm}
\usepackage{bm}
\usepackage{mathrsfs}

% --- DESIGN PACKAGES ---
\usepackage[explicit]{titlesec}
\usepackage{xcolor}
\usepackage{enumitem}

% Standard Academic Colors
\definecolor{academicBlue}{RGB}{0, 50, 100}

% Theorem Environments
\theoremstyle{plain}
\newtheorem{theorem}{Theorem}[section]
\newtheorem{proposition}[theorem]{Proposition}
\newtheorem{lemma}[theorem]{Lemma}
\newtheorem{corollary}[theorem]{Corollary}

\theoremstyle{definition}
\newtheorem{definition}[theorem]{Definition}
\newtheorem{remark}[theorem]{Remark}
\newtheorem{example}[theorem]{Example}

% Section Styling
\titleformat{\section}
  {\normalfont\Large\bfseries\color{academicBlue}}
  {\thesection}{1em}{#1}

\titleformat{\subsection}
  {\normalfont\large\bfseries\color{academicBlue}}
  {\thesubsection}{1em}{#1}

% --- HYPERREF ---
\usepackage[colorlinks=true, linkcolor=academicBlue, citecolor=academicBlue, urlcolor=academicBlue]{hyperref}

% --- DOCUMENT INFO ---
\title{
    \vspace{1cm}
    \huge \textbf{Schemes over $\mathbb{F}_1$ and the Combinatorial Approach to the Riemann Hypothesis} \\
    \vspace{0.5cm}
    \large \textit{A Theoretical Framework based on Monoid Extensions}
}

\author{\textbf{M-I3reak} \\ Institute for Advanced Arithmetic Studies}
\date{\today}

\begin{document}

\maketitle

\begin{abstract}
\noindent Following the program initiated by Tits and Manin, we investigate the geometry of schemes over the "field with one element" ($\mathbb{F}_1$). This paper explores the structural hypothesis that the ring of integers $\mathbb{Z}$ can be realized as a curve over $\mathbb{F}_1$, with the Arakelov compactification arising naturally from base extension to the Archimedean place. We discuss how Kurokawa's formulation of zeta functions for $\mathbb{F}_1$-schemes suggests a spectral interpretation of the zeros of the Riemann zeta function via the unitarity of the Frobenius action on the underlying combinatorial structures.
\end{abstract}

\section{Introduction}

The analogy between number fields and function fields over finite constants remains one of the most fertile grounds in arithmetic geometry. A key obstruction in transferring Weil's proof of the Riemann Hypothesis for curves over finite fields to $\text{Spec}(\mathbb{Z})$ is the absence of a base field. The theory of $\mathbb{F}_1$ proposes that $\mathbb{Z}$ is an algebra over a combinatorial base, denoted $\mathbb{F}_1$.

This paper reviews the construction of $\text{Spec}(\mathbb{Z})$ as an object over $\mathbb{F}_1$ and examines the consequences for the spectral properties of the Frobenius operator.

\section{The Geometry of Monoids}

\subsection{Monoid Schemes}
Since $\mathbb{F}_1$ is not a field in the classical sense, geometry over $\mathbb{F}_1$ is constructed via the theory of monoids.
\begin{definition}[$\mathbb{F}_1$-Scheme]
A scheme over $\mathbb{F}_1$ is essentially a monoid scheme (or a generalized Deitmar scheme), where affine charts are spectra of monoids rather than rings.
\end{definition}
The base extension to $\mathbb{Z}$ is given by the functor $\cdot \otimes_{\mathbb{F}_1} \mathbb{Z}$, which corresponds to the monoid ring construction $\mathbb{Z}[M]$.

\subsection{Arakelov Compactification as Base Change}
A central insight in this framework is the interpretation of the Arakelov compactification.
\begin{theorem}[Manin's Scaling Principle]
The Arakelov surface $\overline{\text{Spec}(\mathbb{Z})}$ can be modeled as a fiber product in the category of generalized schemes:
\begin{equation}
\overline{\text{Spec}(\mathbb{Z})} \sim \text{Spec}(\mathbb{Z}) \times_{\text{Spec}(\mathbb{F}_1)} \text{Spec}(\mathbb{R}).
\end{equation}
\end{theorem}
This suggests that the "point at infinity" corresponds to the base change to the real monoid, providing a geometric reason for the completion of the arithmetic curve.

\section{Cohomological Interpretation}

\subsection{Combinatorial Cohomology}
For a scheme $X_{\mathbb{F}_1}$, the cohomology is combinatorial in nature. The zeta function of an $\mathbb{F}_1$-scheme is typically rational and can be expressed in terms of the Euler characteristic.
\begin{equation}
    \zeta_{X_{\mathbb{F}_1}}(s) = \prod_{k} (s-k)^{-e_k}.
\end{equation}

\subsection{Unitarity of the Frobenius Action}
Kurokawa proposed that the zeta function can be viewed as the characteristic polynomial of a Frobenius operator acting on the cohomology.
\begin{theorem}[Kurokawa's Unitary Formulation]
If $X$ is a scheme over $\mathbb{F}_1$ associated with a lattice or a permutation action, the associated Frobenius operator $\Theta$ is an element of the unitary group (or permutation group).
\end{theorem}

This implies that for objects strictly defined over $\mathbb{F}_1$, the eigenvalues of the Frobenius operator lie on the unit circle (or satisfy a trivial analogue of the Riemann Hypothesis).

\section{Implications for the Riemann Hypothesis}

The transition from $\mathbb{F}_1$ to $\mathbb{Z}$ involves a "quantization" or deformation process.
\begin{enumerate}
    \item \textbf{Lifting Principle:} If $\text{Spec}(\mathbb{Z})$ is indeed a curve over $\mathbb{F}_1$, one expects the cohomological properties of the base to lift to the cover.
    \item \textbf{Spectral Continuity:} The conjecture posits that the unitarity of the combinatorial Frobenius over $\mathbb{F}_1$ induces the critical line property for the zeros of $\zeta(s)$ via a spectral deformation argument.
\end{enumerate}

\begin{corollary}[Conditional Result]
If a suitable cohomology theory exists such that the base change functor $\cdot \otimes_{\mathbb{F}_1} \mathbb{Z}$ preserves the unitarity of the Frobenius action, then the zeros of the Riemann zeta function would lie on the critical line.
\end{corollary}

\section{Concluding Remarks}
The $\mathbb{F}_1$ approach replaces the search for a physical "arithmetic fluid" with a structural search for the combinatorial underpinnings of ring theory. While not yet a complete proof, this framework suggests that the Riemann Hypothesis is a consequence of the rigid combinatorial geometry of the "field with one element."

\end{document}
