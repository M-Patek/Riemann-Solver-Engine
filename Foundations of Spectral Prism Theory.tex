\documentclass[11pt, a4paper]{article}

% --- UNIVERSAL PREAMBLE BLOCK ---
\usepackage[a4paper, top=2.5cm, bottom=2.5cm, left=2.5cm, right=2.5cm]{geometry}
\usepackage{fontspec}
\usepackage[english, bidi=basic, provide=*]{babel}

% Font Setup: Using Noto Sans as default for a modern, technical feel
\babelfont{rm}{Noto Sans}
\babelfont{sf}{Noto Sans}

% --- MATH PACKAGES ---
\usepackage{amsmath}
\usepackage{amssymb}
\usepackage{amsthm}
\usepackage{bm}

% --- DESIGN PACKAGES ---
\usepackage[explicit]{titlesec}
\usepackage{xcolor}
\usepackage{mdframed}
\usepackage{enumitem}
\usepackage{tikz}

% Define custom colors for the OMEGA classification
\definecolor{omegaRed}{RGB}{140, 0, 0}
\definecolor{theoryBlue}{RGB}{0, 51, 102}

% Custom Theorem Styles
\newtheoremstyle{mystyle}%
  {10pt}{10pt}{\itshape}{}{\bfseries\color{theoryBlue}}{.}{.5em}{}
\theoremstyle{mystyle}
\newtheorem{axiom}{Axiom}
\newtheorem{theorem}{Theorem}

% Custom Section Styling
\titleformat{\section}
  {\normalfont\Large\bfseries\color{theoryBlue}}
  {\thesection}{1em}{#1}[\titlerule]

% --- HYPERREF (Must be last) ---
\usepackage[colorlinks=true, linkcolor=theoryBlue, citecolor=omegaRed, urlcolor=theoryBlue]{hyperref}

% --- DOCUMENT INFO ---
\title{
    \vspace{-2cm}
    \begin{flushright}
        \small\textbf{Classification:} \textcolor{omegaRed}{OMEGA (Theoretical Absolutes)} \\
        \small\textbf{Source:} Base Model Simulation Unit (Logic Entity 7-Alpha)
    \end{flushright}
    \vspace{1cm}
    \hrule height 2pt \vspace{0.5cm}
    \huge \textbf{Foundations of Spectral Prism Theory} \\
    \large \textit{Construction of the Three Axioms and Proof of Existential Necessity}
    \vspace{0.5cm} \hrule height 1pt
}
\author{}
\date{}

\begin{document}

\maketitle
\vspace{-2cm}

\begin{mdframed}[linecolor=theoryBlue, linewidth=1pt, roundcorner=5pt, backgroundcolor=gray!5]
\textbf{Abstract:} 
This document provides a rigorous mathematical construction for the three core axioms of \textbf{Grand Unified Arithmetic (GUA)}. We demonstrate that these axioms are not a priori assumptions but geometric necessities arising from the fusion of classical ZFC set theory and modern moduli space theory. We show that if an arithmetic system admits conservation laws (Product Formula), the \textit{Spectral Prism} must exist; if it admits a minimal energy state, the \textbf{Riemann Hypothesis} must hold.
\end{mdframed}

\section{Axiom I: The Ontological Prism}

\begin{axiom}[Ontological Prism]
The ring of integers $\mathbb{Z}$ and the field of real numbers $\mathbb{R}$ are orthogonal sections of the same compact spectral manifold $\mathfrak{S}$.
\end{axiom}

\subsection{Defects in $\text{Spec}(\mathbb{Z})$}
In standard Grothendieck algebraic geometry, $\text{Spec}(\mathbb{Z})$ is an open curve (affine scheme). Much like $\mathbb{C}$ lacks a point at infinity, $\text{Spec}(\mathbb{Z})$ suffers from topological "leakage." This is the fundamental reason why classical harmonic analysis requires compactification for number-theoretic applications.

\subsection{Constructive Proof: Arakelov Compactification}
To close the arithmetic system, $\mathbb{R}$ must be appended as the point at infinity.
\begin{enumerate}[label=\textbf{Step \Alph*:}]
    \item \textbf{Valuation Completion:} By Ostrowski's Theorem, all non-trivial metrics on $\mathbb{Q}$ are either $p$-adic or the Archimedean metric (Real, denoted $\infty$).
    \item \textbf{Conservation of Flux:} For any $x \in \mathbb{Q}^*$, the Product Formula holds:
    $$\prod_{v \in \{p\} \cup \{\infty\}} |x|_v = 1$$
    This formula is the arithmetic realization of \textbf{Stokes' Theorem}. Without the $\infty$ term, the "arithmetic fluid" is not divergence-free.
    \item \textbf{Geometric Result:} The resulting manifold $\mathfrak{S} = \overline{\text{Spec}(\mathbb{Z})}$ is a closed surface where integers and reals are unified.
\end{enumerate}
\textit{Conclusion:} Excluding $\mathbb{R}$ from number theory is topologically illegal; it is equivalent to drawing an open circle.

\section{Axiom II: Dynamical Stability}

\begin{axiom}[Stability]
The arithmetic universe $\mathfrak{S}$ is polystable and possesses a unique Hermitian-Einstein metric.
\end{axiom}

\subsection{Energy Functional and the Ground State}
In arithmetic physics, the energy of the system is described by the Theta functional of a lattice $\overline{E} = (\Lambda, \|\cdot\|)$:
$$\Theta_{\overline{E}}(s) = \sum_{v \in \Lambda \setminus \{0\}} e^{-\pi s \|v\|^2}$$
The Riemann Hypothesis states that the system is in its minimal energy state.

\subsection{The Hitchin-Kobayashi Correspondence}
\begin{enumerate}[label=\textbf{Step \Alph*:}]
    \item \textbf{Convexity:} Grayson and Stuhler proved that the $\Theta$ functional is strictly convex on the space $SL(n, \mathbb{R})/SO(n)$.
    \item \textbf{Stability $\iff$ Minima:} By the Kempf-Ness principle, stability is equivalent to the existence of a zero for the moment map, which corresponds to the global minimum of the energy functional.
    \item \textbf{Uniqueness:} Since $\mathbb{Z}$ is a PID, the standard lattice is semistable. Convexity ensures the existence of a unique metric $\|\cdot\|_{HE}$, the \textbf{Hermitian-Einstein metric}.
\end{enumerate}
\textit{Conclusion:} Stability is not an assumption, but a physical fact dictated by the convex geometry of the lattice.

\section{Axiom III: Quantum Purity}

\begin{axiom}[Unitary Frobenius]
The Frobenius operator acting on the arithmetic manifold $\mathfrak{S}$ is unitary, with its eigenvalue moduli locked to specific weights.
\end{axiom}

\subsection{Derivation of Unitarity}
We examine why the critical line $\text{Re}(s)=1/2$ is a geometric necessity.
\begin{enumerate}[label=\textbf{Step \Alph*:}]
    \item \textbf{Isometry:} Since Axiom II proves the system is in a unique ground state, any natural evolution (the Frobenius Flow $\phi^t$) must preserve this state.
    \item \textbf{Operator Rigidity:} An isometry in Hilbert space corresponds to a \textbf{Unitary Operator} $U$.
    \item \textbf{Spectral Radius:} For any unitary $U$, $|\lambda| = 1$. In the context of the cohomology $H^k$ of an arithmetic curve (dim $d=1$), Motives theory dictates a pure weight of $k$.
    \item \textbf{Normalization:} For $H^1$, the weight is 1, thus the actual eigenvalue modulus is scaled: $|\lambda| = q^{1/2}$.
\end{enumerate}
\textit{Conclusion:} The Riemann Hypothesis is the projection of the unitarity of the Frobenius flow onto the complex plane.

\section*{Summary: The Logical Closure}
The three axioms form a self-consistent universe:
\begin{itemize}
    \item \textbf{Geometry (I):} Establishes the closed container for arithmetic flux.
    \item \textbf{Dynamics (II):} Forces the fluid into a unique stable ground state.
    \item \textbf{Spectra (III):} Ensures evolution remains unitary and conservative.
\end{itemize}
Within Grand Unified Arithmetic, the Riemann Hypothesis is an absolute structural truth.

\end{document}
